\documentclass[a4paper,11pt]{article}
\usepackage[utf8]{inputenc}
\usepackage[T1]{fontenc}
\usepackage{amsmath,amssymb,bm}
\usepackage{geometry}
\geometry{margin=2.5cm}
\usepackage[french]{babel}
\usepackage{graphicx}
\usepackage{float}
\usepackage{hyperref}
\usepackage{lmodern}
\usepackage{setspace}
\usepackage{titlesec}
\usepackage[table]{xcolor}
\usepackage{colortbl}
\usepackage{gensymb}
\usepackage{subcaption}
\usepackage{csquotes}
\usepackage{booktabs}
\usepackage{mdframed}




\begin{document}

\begin{titlepage}

\hrule

\begin{figure}[h]
    \centering
    \begin{subfigure}[b]{0.3\textwidth}
        \includegraphics[width=\textwidth]{/Users/paulcailleu/Documents/Etudes/ENSTA/optimisation_discrete/images/ipp_logo.png}
    \end{subfigure}
    \hspace{1cm}
    \begin{subfigure}[b]{0.4\textwidth}
        \includegraphics[width=\textwidth]{/Users/paulcailleu/Documents/Etudes/ENSTA/optimisation_discrete/images/Logo_ENSTA-2025.png}
    \end{subfigure}
\end{figure}
\hrule

\vspace*{1cm}

\begin{center}

    \huge Samuel Bergina \& Paul Cailleu\\

    \vspace{0.2cm}

    \Large \textit{Cycle Ingénieur ENSTA Paris, Énergies en Transition}\\

    \vspace{1.2cm}

    \hrule

    \vspace{0.5cm}

    {\LARGE \textbf{Rapport Scientifique}}\\

    \vspace{0.5cm}

    {\LARGE Production optimale d'un ensemble d'unité de production couplé à un réseau de stockage hydraulique}\\

    \vspace{0.5cm}

    \hrule

    \vspace{1cm}
    \textbf{Projet d'optimisation discrète}\\
    \vspace{1cm}
    {\Large \textit{Décembre 2025 -- Février 2026}}\\
\end{center}

\end{titlepage}

\setcounter{figure}{0}

\newpage
\tableofcontents

\newpage



\section{Formulation MILP UC hydro-thermique}

Le problème de \textit{Unit Commitment} (UC) consiste à décider, à chaque
pas de temps, quelles unités de production sont en marche (\emph{commitment})
et à quelle puissance elles fonctionnent (\emph{dispatch}), de façon à
satisfaire la demande au coût minimum tout en respectant les contraintes
techniques (rampes, durées minimales de marche/arrêt, bilans hydrauliques).
La formulation est un programme mixte en nombres entiers (MILP) couplant
un parc thermique et une cascade hydraulique.

% -----------------------------------------------------------------------
\subsection{Ensembles}
% -----------------------------------------------------------------------

\begin{itemize}
  \item $\mathcal{T}=\{1,\dots,T\}$ : pas de temps (horizon de planification).
  \item $\mathcal{L}$ : unités thermiques (centrales à flamme, turbines à gaz, etc.).
  \item $\mathcal{K}$ : arcs du réseau hydraulique.
        Chaque arc $k\in\mathcal{K}$ correspond à un groupe
        turbine/pompe reliant un réservoir amont $r^{\rm src}_k$ à un réservoir aval
        $r^{\rm dst}_k$. La topologie est un graphe orienté acyclique (DAG)
        quelconque~: chaîne simple, arcs parallèles et bifurcations sont tous
        supportés.
  \item $\mathcal{R}=\{1,\dots,|\mathcal{K}|+1\}$ : réservoirs.
        Le nœud $r=|\mathcal{K}|+1$ est le \emph{réservoir de décharge}
        (exutoire/rivière aval), modélisé par $V^{\max}=+\infty$ et $V^0=0$.
  \item $\mathcal{S}=\{1,\dots,S\}$ : segments de la linéarisation par morceaux
        de la courbe débit–puissance de chaque turbine.
\end{itemize}

% -----------------------------------------------------------------------
\subsection{Paramètres}
% -----------------------------------------------------------------------

\begin{align*}
& dt &&\text{durée d'un pas de temps} \\
& \tau^{+}_l,\ \tau^{-}_l &&\text{durées minimales de marche / d'arrêt de l'unité } l \\
& d_t &&\text{demande électrique totale au pas $t$ (MW)} \\
& c_{l,t} &&\text{coût marginal variable de l'unité $l$ au pas $t$ (€/MWh)} \\
& su_l &&\text{coût fixe de démarrage de l'unité $l$ (€/démarrage)} \\
& P^{\min}_{l,t},\ P^{\max}_{l,t} &&\text{bornes de puissance thermique de $l$ (MW)} \\
& g_{l,t} &&\text{rampe max de $l$ (MW/h) : variation de puissance autorisée par pas} \\[4pt]
& r^{\rm src}_k,\ r^{\rm dst}_k &&\text{réservoir source / destination de l'arc $k$ (topologie DAG)} \\
& F^{\min +}_{k,t},\ F^{\max +}_{k,t} &&\text{débit turbine min/max sur l'arc $k$ au pas $t$} \\
& g^+_k &&\text{rampe turbine (variation de débit autorisée par pas, en débit/h)} \\
& P^{\min +}_k,\ P^{\max +}_k &&\text{puissance turbine min/max sur l'arc $k$ (MW)} \\
& F^{\min -}_{k,t},\ F^{\max -}_{k,t} &&\text{débit pompe min/max sur l'arc $k$} \\
& g^-_k &&\text{rampe pompe (débit/h)} \\
& \rho_k &&\text{consommation spécifique de pompage (MW/débit)} \\
& f_{k,i},\ P_{k,i} &&\text{points de rupture de la courbe turbine ($i=1\ldots S{+}1$)} \\
& V^{\min}_r,\ V^{\max}_r &&\text{bornes de volume du réservoir $r$} \\
& a_{r,t} &&\text{apport naturel (pluie/fonte) au réservoir $r$ au pas $t$} \\
& V^0_r &&\text{volume initial du réservoir $r$} \\
\end{align*}

% -----------------------------------------------------------------------
\subsection{Variables de décision}
% -----------------------------------------------------------------------

\paragraph{Thermique.}
\begin{align*}
& p_{l,t} \ge 0 &&\text{puissance injectée par l'unité $l$ au pas $t$ (MW)} \\
& y_{l,t} \in \{0,1\} &&\text{1 si l'unité $l$ est en marche au pas $t$, 0 sinon} \\
& u_{l,t} \in \{0,1\} &&\text{1 si $l$ démarre entre $t{-}1$ et $t$} \\
& d_{l,t} \in \{0,1\} &&\text{1 si $l$ s'arrête entre $t{-}1$ et $t$} \\
\end{align*}

\paragraph{Hydraulique.}
\begin{align*}
& f^+_{k,t} \ge 0 &&\text{débit turbiné sur l'arc $k$ au pas $t$} \\
& f^-_{k,t} \ge 0 &&\text{débit pompé sur l'arc $k$ au pas $t$ (sens aval$\to$amont)} \\
& z_{k,t} \in \{0,1\} &&\text{1 si la pompe de l'arc $k$ est en marche au pas $t$} \\
& P^+_{k,t} \ge 0 &&\text{puissance électrique produite par la turbine $k$ (MW)} \\
& z_{k,s,t} \in \{0,1\} &&\text{indicatrice du segment $s$ actif sur l'arc $k$ au pas $t$} \\
& \theta_{k,s,t}\in[0,1] &&\text{position relative dans le segment $s$ (interpolation)} \\
& V_{r,t} \ge 0 &&\text{volume stocké dans le réservoir $r$ au début du pas $t$} \\
\end{align*}

% -----------------------------------------------------------------------
\subsection{Objectif}
% -----------------------------------------------------------------------

On minimise le coût total de production sur l'horizon~:
\[
\min \underbrace{\sum_{t\in\mathcal{T}}\sum_{l\in\mathcal{L}}
      c_{l,t}\, p_{l,t}\, dt}_{\text{coût variable de fonctionnement}}
\;+\;
\underbrace{\sum_{t\in\mathcal{T}}\sum_{l\in\mathcal{L}}
      su_l\, u_{l,t}}_{\text{coût de démarrage}}
\]
La production hydraulique n'a pas de coût variable explicite (l'eau est
gratuite) : elle contribue indirectement en réduisant le recours aux
unités thermiques coûteuses.

% -----------------------------------------------------------------------
\subsection{Contraintes thermiques}
% -----------------------------------------------------------------------

\paragraph{Bornes de puissance.}
Une unité ne produit que si elle est en marche ($y_{l,t}=1$) et doit alors
rester entre sa puissance minimale technique et sa puissance maximale~:
\begin{align}
& p_{l,t} \ge P^{\min}_{l,t}\, y_{l,t} &&\forall l\in\mathcal{L},\, t\in\mathcal{T} \\
& p_{l,t} \le P^{\max}_{l,t}\, y_{l,t} &&\forall l\in\mathcal{L},\, t\in\mathcal{T}
\end{align}

\paragraph{Rampes.}
La variation de puissance entre deux pas consécutifs est limitée par la
rampe physique de l'unité (montée et descente symétriques ici)~:
\begin{align}
& p_{l,t} - p_{l,t-1} \le g_{l,t}\, dt &&\forall l\in\mathcal{L},\, t>1 \\
& p_{l,t-1} - p_{l,t} \le g_{l,t}\, dt &&\forall l\in\mathcal{L},\, t>1
\end{align}

\paragraph{Transitions d'état (démarrage / arrêt).}
La variable binaire $u_{l,t}$ (resp. $d_{l,t}$) vaut 1 exactement quand
$y$ passe de 0 à 1 (resp. de 1 à 0)~; les deux événements s'excluent
mutuellement~:
\begin{align}
& u_{l,t} - d_{l,t} = y_{l,t} - y_{l,t-1} &&\forall l\in\mathcal{L},\, t>1 \\
& u_{l,t} + d_{l,t} \le 1 &&\forall l\in\mathcal{L},\, t\in\mathcal{T}
\end{align}

\paragraph{Durées minimales de marche et d'arrêt.}
Après un démarrage, l'unité doit rester en marche au moins $\tau^+_l$ pas~;
après un arrêt, elle doit rester arrêtée au moins $\tau^-_l$ pas~:
\begin{align}
& \sum_{t'=t}^{t+\tau^+_l-1} y_{l,t'} \ge \tau^+_l\, u_{l,t}
  &&\forall l\in\mathcal{L},\; t :\ t+\tau^+_l-1 \le T \\
& \sum_{t'=t}^{t+\tau^-_l-1} (1-y_{l,t'}) \ge \tau^-_l\, d_{l,t}
  &&\forall l\in\mathcal{L},\; t :\ t+\tau^-_l-1 \le T
\end{align}
Ces contraintes ne sont posées que lorsque la fenêtre $[t, t+\tau-1]$
tient entièrement dans l'horizon.

% -----------------------------------------------------------------------
\subsection{Contraintes hydrauliques}
% -----------------------------------------------------------------------

\paragraph{Bornes de débit turbine.}
Le débit turbiné sur l'arc $k$ est borné à chaque pas~:
\begin{align}
& F^{\min +}_{k,t} \le f^+_{k,t} \le F^{\max +}_{k,t} &&\forall k\in\mathcal{K},\, t\in\mathcal{T}
\end{align}

\paragraph{Rampes turbine.}
Le débit ne peut varier trop vite (cavitation, contraintes mécaniques)~:
\begin{align}
& f^+_{k,t} - f^+_{k,t-1} \le g^+_k\, dt &&\forall k\in\mathcal{K},\, t>1 \\
& f^+_{k,t-1} - f^+_{k,t} \le g^+_k\, dt &&\forall k\in\mathcal{K},\, t>1
\end{align}

\paragraph{Pompe : débit et rampe.}
La pompe ne peut fonctionner que si $z_{k,t}=1$, et son débit est nul
sinon (contrainte de type \textit{on/off})~:
\begin{align}
& F^{\min -}_{k,t}\, z_{k,t} \le f^-_{k,t} \le F^{\max -}_{k,t}\, z_{k,t}
  &&\forall k\in\mathcal{K},\, t\in\mathcal{T} \\
& f^-_{k,t} - f^-_{k,t-1} \le g^-_k\, dt &&\forall k\in\mathcal{K},\, t>1 \\
& f^-_{k,t-1} - f^-_{k,t} \le g^-_k\, dt &&\forall k\in\mathcal{K},\, t>1
\end{align}

\paragraph{Bornes de puissance turbine.}
La puissance produite par la turbine doit rester dans sa plage de
fonctionnement (indépendamment du débit, les bornes sont vérifiées via la
courbe PWLC ci-dessous)~:
\begin{align}
& P^{\min +}_k \le P^+_{k,t} \le P^{\max +}_k &&\forall k\in\mathcal{K},\, t\in\mathcal{T}
\end{align}

% -----------------------------------------------------------------------
\subsection{Linéarisation de la courbe débit–puissance (PWLC)}
% -----------------------------------------------------------------------

La relation $P^+_{k,t} = \varphi_k(f^+_{k,t})$ entre débit turbiné et
puissance produite est non linéaire (rendement variable avec la hauteur de
chute). On l'approche par une fonction linéaire par morceaux à $S$ segments,
définie par $S+1$ points de rupture $(f_{k,i},\, P_{k,i})$.

La technique SOS-2 (\textit{Special Ordered Set}) sélectionne un unique
segment actif $s$ via la variable $z_{k,s,t}$, puis interpole
linéairement à l'intérieur de ce segment via $\theta_{k,s,t}$~:
\begin{align}
& \sum_{s\in\mathcal{S}} z_{k,s,t} = 1
  &&\forall k\in\mathcal{K},\, t\in\mathcal{T}
  \quad\text{(exactement un segment actif)} \\
& \theta_{k,s,t} \le z_{k,s,t}
  &&\forall k\in\mathcal{K},\, s\in\mathcal{S},\, t\in\mathcal{T}
  \quad\text{($\theta$ nul si segment inactif)} \\
& f^+_{k,t} = \sum_{s\in\mathcal{S}} \bigl(f_{k,s}\, z_{k,s,t}
    + (f_{k,s+1}-f_{k,s})\,\theta_{k,s,t}\bigr)
  &&\forall k\in\mathcal{K},\, t\in\mathcal{T} \\
& P^+_{k,t} = \sum_{s\in\mathcal{S}} \bigl(P_{k,s}\, z_{k,s,t}
    + (P_{k,s+1}-P_{k,s})\,\theta_{k,s,t}\bigr)
  &&\forall k\in\mathcal{K},\, t\in\mathcal{T}
\end{align}
Les deux dernières contraintes imposent simultanément $f^+$ et $P^+$ sur
le même segment~: la courbe $P(\!f)$ est ainsi respectée de façon exacte
sur chaque morceau.

% -----------------------------------------------------------------------
\subsection{Bilan volumétrique des réservoirs (topologie DAG générale)}
% -----------------------------------------------------------------------

On définit pour chaque réservoir $r$ les ensembles d'arcs incidents~:
\[
  \mathcal{K}^{\rm in}_r  = \{k\in\mathcal{K} : r^{\rm dst}_k = r\}
  \quad\text{(arcs \emph{arrivant} en }r\text{)},
  \qquad
  \mathcal{K}^{\rm out}_r = \{k\in\mathcal{K} : r^{\rm src}_k = r\}
  \quad\text{(arcs \emph{partant} de }r\text{)}.
\]

Le bilan d'eau à chaque pas de temps s'écrit~:

\begin{equation}
  V_{r,t+1} = V_{r,t}
  \;+\; a_{r,t}\,dt
  \;+\!\!\underbrace{\sum_{k\in\mathcal{K}^{\rm in}_r} \!\!f^+_{k,t}\,dt}_{\substack{\text{eau amenée}\\\text{par turbinage}}}
  \;-\!\!\underbrace{\sum_{k\in\mathcal{K}^{\rm out}_r}\!\!f^+_{k,t}\,dt}_{\substack{\text{eau prélevée}\\\text{par turbinage}}}
  \;-\!\!\underbrace{\sum_{k\in\mathcal{K}^{\rm in}_r} \!\!f^-_{k,t}\,dt}_{\substack{\text{eau remontée}\\\text{par pompage}}}
  \;+\!\!\underbrace{\sum_{k\in\mathcal{K}^{\rm out}_r}\!\!f^-_{k,t}\,dt}_{\substack{\text{eau renvoyée}\\\text{par pompage}}}
  \quad \forall r\in\mathcal{R},\; t < T
\end{equation}

\noindent\textbf{Lecture intuitive.}
Le turbinage sur un arc $k$ \emph{entrant} en $r$ remplit le réservoir~;
le turbinage sur un arc $k$ \emph{sortant} le vide.
Le pompage inverse le sens~: sur un arc entrant en $r$, la pompe
\emph{reprend} de l'eau dans $r$ pour la renvoyer en amont~;
sur un arc sortant de $r$, elle \emph{rapporte} de l'eau depuis l'aval.
La convention $\sum_\emptyset(\cdot)=0$ gère naturellement les nœuds
sources (aucun arc entrant) et le nœud de décharge (aucun arc sortant)
sans condition particulière.

\noindent\textbf{Bornes et condition initiale~:}
\[
  V^{\min}_r \le V_{r,t} \le V^{\max}_r \quad \forall r,t,
  \qquad
  V_{r,1} = V^0_r \quad \forall r.
\]

\paragraph{Cas particulier : chaîne simple.}
Si $r^{\rm src}_k = k$ et $r^{\rm dst}_k = k+1$ pour tout $k$
(cascade linéaire), alors $\mathcal{K}^{\rm in}_r = \{r-1\}$ et
$\mathcal{K}^{\rm out}_r = \{r\}$, et l'équation~(23) se réduit à~:
\[
  V_{r,t+1} = V_{r,t} + a_{r,t}\,dt
  + f^+_{r-1,t}\,dt - f^+_{r,t}\,dt
  - f^-_{r-1,t}\,dt + f^-_{r,t}\,dt.
\]

% -----------------------------------------------------------------------
\subsection{Bilan puissance (équilibre offre–demande)}
% -----------------------------------------------------------------------

À chaque pas de temps, la somme des injections (thermique + hydro)
moins la consommation des pompes doit exactement couvrir la demande~:
\begin{equation}
\underbrace{\sum_{l\in\mathcal{L}} p_{l,t}}_{\text{thermique}}
 + \underbrace{\sum_{k\in\mathcal{K}} P^+_{k,t}}_{\text{hydro}}
 - \underbrace{\sum_{k\in\mathcal{K}} \rho_k\, f^-_{k,t}}_{\text{pompage (conso)}}
 = d_t \quad \forall t\in\mathcal{T}
\end{equation}
Le pompage est une charge électrique~: il consomme $\rho_k\,f^-_{k,t}$
MW sur l'arc $k$ pour refouler du débit $f^-_{k,t}$ en amont.



Lien du repo github : \href{https://github.com/PaulCailleu/Optimisation_discrete}{$github.com/PaulCailleu/Optimisation\_discrete$}

\end{document}
